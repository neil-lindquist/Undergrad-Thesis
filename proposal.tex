\documentclass[titlepage]{article}

\usepackage{algorithmic}
\usepackage{algorithm}
\usepackage{amsmath}
\usepackage{etoolbox}
\usepackage{hyperref}
\usepackage{mathtools}


%number the bib as a section
\patchcmd{\thebibliography}{\section*}{\section}{}{}
\renewcommand\refname{Preliminary References}

\begin{document}

\title{
	Compression of Linear Algebra Data Structures to Reduce Memory Access Latencies \\
	\bigskip
	\Large Thesis Proposal}
\author{Neil Lindquist}
\date{October 15th, 2018}

\maketitle

\begin{center}
	{\Large Compression of Linear Algebra Data Structures to Reduce Memory Access Latencies}
	
	Neil Lindquist
	
	\bigskip
	\bigskip
	
	\textbf{Approved By:}
	
	\bigskip
	
	Mike Heroux\\
	\textit{Thesis Advisor}\\
	\textit{Scientist in Residence}

	\bigskip

	%TODO find reader 1
	Bob Hesse\\
	\textit{Faculty Reader}\\
	\textit{Associate Professor of Mathematics}

	\bigskip

	%TODO find reader 2
	Jeremy Iverson\\
	\textit{Faculty Reader}\\
	\textit{Assistant Professor of Computer Science}
	
	\bigskip
	
	Imad Rahal\\
	\textit{Chair, Department of Computer Science}
	
	\bigskip
	
	Bret Benesh\\
	\textit{Chair, Deparment of Mathematics}
\end{center}

\clearpage


\section{Statement of Purpose}
The goal of this project is to improve the performance of certain parts
used in some scientific computations.
These calculations include problems such as modeling fluid flow, chemical processes and electromagnetism.
The specific parts being optimized are iterative solvers for sparse, linear equations.
The majority of the time spent in these types of solvers is spent moving data from main memory
to the processor where the actual arithmetic is done~\cite{Lawlor:2013:compression}.
So, by compressing the largest pieces of this data, this project will try to reduce the time spent waiting and thus improve the overall efficiency.
This improvement can allow running important computations more often and with more detail.

As a more detailed description of the computations being improved, a basic description of
this type of linear solver follows.
The specific problem this work focuses on is solving systems of linear equations.
A simple example of a system of linear equations is finding values for \(x\) and \(y\) such that
\begin{align*}
	5x + 3y &= 10\\
	3x + 3y &= 6
\end{align*}
are both true.
Some of the linear system used in real-work problems, such as representing a set of physics equations,
can become very large, with possibly hundreds of millions of equations~\cite{Davis:2011:FloridaMatrixCollection}.
Not only does this large size result in long times to solve the problem,
it means that the problem has to be stored in the slow-to-access main memory instead of the quick-to-access caches.
Accessing main memory can take over 100 times as long as a single arithmetic operation~\cite{Goumas:2009:performanceEval}.
So, reductions in the amount of memory used to store key components of the problem
have the opportunity to provide significant improvements in performance, even if it increases the actual calculations needed.
Specifically, data compression schemes with very simple decoding algorithms are being experimented with
to find the most effective way to store the key data structures of the solvers.


\section{Proposal Summary}

This project is working at finding compression methods that improve performance by reducing memory access latencies.
The High Performance Conjugate Gradient (HPCG) benchmark is being used as the codebase to test compression methods with~\cite{Dongarra:2015:HPCG}.
This code is modified to utilize different compression methods, then is used to test the compression's performance.
Because HPCG is a benchmark that uses a sparse, iterative linear solver to test performance,
it provides both an implementation of a sparse, iterative linear solver (specificly Conjugate Gradient)
and measures of performance for both the overall solver as well as the major components.
Compression techniques being used include single precision floats, SZ compression and ZFP compression for floating point data
and Elias Gamma coding, Elias Delta coding, SZ compression, Huffman coding for integers~\cite{Di:2016:SZ,Lindstrom:2014:zfp,Elias:1975:codeword,Huffman:1952:coding}.


\section{Preliminary Outline}

At the highest level, the thesis will have a vary simple outline.
First, it will provide enough information to understand the goals, test problem, compression methods used and the interactions there of.
Next will be the actual test results and analysis there of.
Finally, the paper will present any conclusions discovered in the test results as well as ways the work can be extended in the future.
The first section will cover a number of relevant topics.
Firstly, the relevance and goal of this work will be presented.
Next, the current state of the art for this type of computations will be described,
including both the computational aspects and the underlying mathematics.
After that will be a description of the code base used to create the test results.
Then, a short discussion of restrictions on the compression and data access patterns will be provided.
Finally, information on the various compression methods will be provided.
The other two overall sections will be simpler.
The test results will largely contain tables of times and other metrics, but will also contain analysis of the results.
The concluding section will summarize the results and conclusions of the work as well as propose future projects that builds on these results.


\nocite{*}
\bibliographystyle{plain}
\bibliography{bibliograph}

\end{document}