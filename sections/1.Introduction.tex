Solving large, sparse linear systems of equations plays an important role in certain scientific computations.
For example, the finite element method solves a system of linear equations in order to approximate the solution to certain partial differential equations~\cite{Saad:2003:IterativeMethods}.
These problems can be large, with possibly millions of variables~\cite{Davis:2011:FloridaMatrixCollection}.
So, solving these problems efficiently requires a fast linear solver.

The particular class of linear systems being looked at are sparse, meaning that they have a high proportion of zero coefficients.
Iterative solvers are often used to solve these large, sparse systems.
These solvers take an initial guess then improve it until it is within some tolerance~\cite{Saad:2003:IterativeMethods}.
On modern computers, these solvers often spend significantly most of their time fetching data from main memory to the processor where the actual computation is done~\cite{Lawlor:2013:compression}.
This work tries to improve the performance of these types of solvers by compressing the data to reduce the time spent accessing main memory.

\subsection{Previous Work}
Much work has been done on various aspects of utilizing single precision floating point numbers, while retaining the accuracy of double precision numbers.
One such approach uses the fact that iterative solvers can take an initial guess of the solution to jump start progress.
So, many iterations can be done at the cheaper single precision, then double precision iterations refine the solution to sufficient accuracy~\cite{Babolin:2008:coursePass, Buttari:2007:coursePass}.
Another main approach is to applying the preconditioner using single precision, while otherwise using double precision, which result in similar accuracy unless the matrix is poorly conditioned~\cite{Buttari:2008:mixedPrec, Hogg:2010:multiplePasses}.

Another effort at compressing large, sparse Linear Systems is Compressed Column Index (CCI) format to store matrices~\cite{Lawlor:2013:compression}.
This format is based of Compressed Sparse Row (CSR) matrix format except uses a compressed representation of the column indices.
This project generalizes CCI matrix format by trying compression with additional data structures and by trying additional compression schemes.