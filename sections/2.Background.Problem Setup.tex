%problem and discresionization
The problem used to create the linear system used by HPCG, and thus by this project, is a three dimensional partial differential equation (PDE) model~\cite{Dongarra:2015:HPCG}.
This problem is approximating the function \(u(x, y, z)\) over the three dimensional rectangular region \(\Omega\in\mathbb{R}^3\) such that
\[
	\Delta u = \frac{\partial^2 u}{\partial x^2} + \frac{\partial^2 u}{\partial y^2} + \frac{\partial^2 u}{\partial z^2} = 0,
\] with \(u(x, y, z) = 1\) along the boundaries of \(\Omega\).
Note that the solution is \(u(x, y, z) = 1\) over \(\Omega\).
The linear system is created by using the finite difference method with a 27-point stencil on the PDE over a rectangular grid with nodes of fixed distance.
The matrix's diagonal consists of the value 26, and -1's fill the entries for the row's 26 grid neighbors.
The right hand side of the equation has a value of 14 for corner points, 12 for edge points, 9 for side points and 0 for interior points~\cite{Kincaid:2009:Numerical}.
The solution vector consists of all 1's.
Finally, the zero vector is used as the inital guess.

