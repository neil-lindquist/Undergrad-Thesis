%problem and discresionization
The problem used to create the linear system used by HPCG, and thus by this project, is a three dimensional partial differential equation (PDE) model~\cite{Dongarra:2015:HPCG}.
This problem is approximating the function \(u(x, y, z)\) over the three dimensional rectangular region \(\Omega\in\mathbb{R}^3\) such that
\[
	\Delta u = \frac{\partial^2 u}{\partial x^2} + \frac{\partial^2 u}{\partial y^2} + \frac{\partial^2 u}{\partial z^2} = 0,
\] with \(u(x, y, z) = 1\) along the boundaries of \(\Omega\).
Note that the solution is \(u(x, y, z) = 1\) over \(\Omega\).
The linear system is created by using the finite difference method with a 27-point stencil on the PDE over a rectangular grid with nodes of fixed distance.
The matrix's diagonal consists of the value 26, and -1's fill the entries for the row's 26 grid neighbors.
%TODO should this be described in more detail, sum of the adjacent points on the boundary
The right hand side of the equation has a value of 14 for corner points, 12 for edge points, 9 for side points and 0 for interior points~\cite{Kincaid:2009:Numerical}.
The solution vector consists of all 1's.

%TODO discuss MPI setup

% HPCG's solve details
HPCG uses an implementation of Conjugate Gradient algorithm with a multigrid preconditioner variant~\cite{Dongarra:2015:HPCG}.
As HPCG is designed to emulate the performance characteristics of real world problems with out needing to be a robust solver, it only uses 3 levels of grid coarseness with only a since smoother pass at the coarsest grid level.
%TODO does this note need to be exteneded
The smoother used by the multigrid is based on a symmetric Gauss-Seidel step, however each process uses the old value for entries located on other processes.
%TODO improve wording
The restriction operation simply samples half the points in dimension, resulting in a reduction of grid size by a factor of eight in each level of coarseness.
To prolong the coarse grids, each coarse point is added to the fine point in was sampled from.
The zero vector is used as the overall initial guess for \(x\), as well as the initial guess for each grid level in the multigrid cycle.

