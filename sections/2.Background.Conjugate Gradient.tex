Conjugate Gradient is the iterative solver used by HPCG~\cite{Dongarra:2015:HPCG}.
Symmetric, positive definite matrices will guarantee the converge of Conjugate Gradient to the correct solution within \(n\) iterations, where \(n\) is the number of dimensions, when using exact algebra~\cite{Saad:2003:IterativeMethods}.
More importantly, Conjugate Gradient can be used as in iterative method, providing a solution, \(\vec{x}\), where \(\left\|\mat{A}\vec{x}-\vec{b}\right\|\) is within some tolerance, \(\epsilon\), after significantly fewer than \(n\) iterations, allowing it to find solutions to problems where even \(n\) iterations is infeasible~\cite{Shewchuk:1994:IntroToCG}.
%TODO go into more high level details of CG

%Steepest Decent
To understand the Conjugate Gradient, first consider the quadratic form of \(\mat{A}\vec{x} = \vec{b}\).
The quadratic form is a function \(f:\mathbb{R}^n\to\mathbb{R}\) where
\begin{equation}
\label{eq:cg-quad-form}
	f(\vec{x}) = \tfrac{1}{2}\vec{x}^T\mat{A}\vec{x} - \vec{b}\cdot\vec{x} + c
\end{equation}
for some \(c\in\mathbb{R}\).
Note that
\[
	\nabla f\left(\vec{x}\right) = \tfrac{1}{2}\left(\mat{A}+\mat{A}^T\right)\vec{x}-\vec{b}
\]
Then, when \(\mat{A}\) is symmetric,
\begin{equation*}
	\nabla f(\vec{x}) = \mat{A}\vec{x} - \vec{b}
\end{equation*}
So, the solution to \(\mat{A}\vec{x} = \vec{b}\) is the sole critical point of \(f\)~\cite{Nearing:2010:toolsForPhysics}.
Since \(\mat{A}\) is the Hessian matrix of \(f\) at the point, if \(\mat{A}\) is positive definite, then that critical point is a minimum.
Thus, if \(\mat{A}\) is a symmetric, positive definite matrix, then the minimum of \(f\) is the solution to \(\mat{A}\vec{x} = \vec{b}\)~\cite{Shewchuk:1994:IntroToCG}.

The method of Steepest Decent is useful for understanding Conjugate Gradient, because they both use a similar approach to minimize Equation~\ref{eq:cg-quad-form}, and thus solve \(\mat{A}\vec{x}=\vec{b}\).
This shared approach is to take an initial \(\vec{x}_0\) and move downwards in the steepest direction, within certain constraints, of the surface defined by Equation~\ref{eq:cg-quad-form}~\cite{Nearing:2010:toolsForPhysics}.
Because the gradient at a point is the direction of maximal increase, \(\vec{x}\) should be moved in the opposite direction of the gradient.
Thus, to compute the next value of \(\vec{x}\), use
\begin{equation}
\label{eq:cg-xUpdate}
	\vec{x}_{i+1} = \vec{x}_i + \alpha_i \vec{r}_i
\end{equation}
for some \(\alpha_i > 0\) and where \(\vec{r}_i = -\nabla f\left(\vec{x}_i\right) = \vec{b} - \mat{A}\vec{x}_i\) is the residual of \(\vec{x}_i\).
Since \(\mat{A}\vec{x} = \vec{b}\) is the only critical point and a minimum of the quadratic function, \(f\), the ideal value of \(\alpha_i\) is the one that minimizes \(f\left(\vec{x}_{i+1}\right)\).
Thus, choose \(\alpha_i\) such that
\begin{align*}
	0
	&= \tfrac{\mathrm{d}}{\mathrm{d} \alpha_i} f\left(\vec{x}_{i+1}\right) \\
	&= \tfrac{\mathrm{d}}{\mathrm{d} \alpha_i} f\left(\vec{x}_i + \alpha \vec{r}_i\right) \\
%
	\alpha_i
	&= \frac{\vec{r}_i\cdot \vec{r}_i}{\vec{r}_i\cdot\mat{A}\vec{r}_i}.
\end{align*}
Note that by using Equation~\ref{eq:cg-xUpdate}, we can derive
\begin{equation}
\label{eq:CG-improvedResidual}
	\vec{r}_{i+1} = \vec{r}_i - \alpha\mat{A}\vec{r}_i.
\end{equation}
Because \(\mat{A}\vec{r}_i\) is already computed to find \(\alpha_i\), using Equation~\ref{eq:CG-improvedResidual} to compute the residual results in one less matrix-vector product per iteration.
Algorithm~\ref{alg:bg-cg-alg-steepestDecent} shows the resulting algorithm.
\begin{algorithm}[tb]
	\begin{algorithmic}
		\STATE \(\vec{r}_0 \gets \vec{b} - \mat{A}\vec{x}_0\)
		\FOR{\(i = 0, 1, \dots\) \textbf{until} \(\left\|\vec{r}_i\right\| \leq \epsilon\)}
			\STATE \(\alpha_i \gets \frac{\vec{r}_i\cdot \vec{r}_i}{\vec{r}_i\cdot\mat{A}\vec{r}_i}\)
			\STATE \(\vec{x}_{i+1} = \vec{x}_i + \alpha_i\vec{r}_i\)
			\STATE \(\vec{r}_{i+1} = \vec{r}_i - \alpha\mat{A}\vec{r}_i\)
		\ENDFOR
	\end{algorithmic}
	
	\caption{Steepest Decent~\cite{Shewchuk:1994:IntroToCG}.}
	\label{alg:bg-cg-alg-steepestDecent}
\end{algorithm}


\begin{example}
\label{ex:CG-SteepestDecent}
%x = (2, 1)
Consider the linear system
\[
	\mat{A} = \begin{bmatrix}
		2 & 1 \\
		1 & 3
	\end{bmatrix},
	\quad
	\vec{b} = \begin{bmatrix}
		5 \\
		5 
	\end{bmatrix}
\]
and use \(c=0\).
Note that the solution is
\[
	\vec{x} = \begin{bmatrix}
		2 \\ 1
	\end{bmatrix}.
\]
When starting at the origin, the iteration of Method of Steepest Decent becomes
\begin{align*}
	 \vec{x}_0 &= \begin{bmatrix}0\\0\end{bmatrix}
	&\vec{r}_0 &= \begin{bmatrix}5\\5\end{bmatrix}
	&\alpha_0  &= 2/7 \\
	%
	 \vec{x}_1 &= \begin{bmatrix}10/7\\10/7\end{bmatrix}
	&\vec{r}_1 &= \begin{bmatrix}5/7\\-5/7\end{bmatrix}
	&\alpha_1  &= 2/3 \\
	%
	 \vec{x}_2 &= \begin{bmatrix}40/21\\20/21\end{bmatrix}
	&\vec{r}_2 &= \begin{bmatrix}5/21\\5/21\end{bmatrix}
	&\alpha_2  &= 2/7 \\
	%
	 \vec{x}_3 &= \begin{bmatrix}290/147\\50/49\end{bmatrix}
	&\vec{r}_3 &= \begin{bmatrix}5/147\\-5/147\end{bmatrix}
	&\alpha_3  &= 2/3 \\
	%
	&\vdots &&\vdots &&\vdots
\end{align*}
The \(\vec{x}_i\)'s are plotted with a contour graph of the quadratic form in Figure~\ref{fig:CG-steepestDecentExample-Iterations}. \qed
\begin{figure}
	\centering
	\ifdraft{}{
		\begin{tikzpicture}
			\begin{axis} [
				xlabel=\(x_1\),
				ylabel=\(x_2\),
				view={0}{90},
			]
				\addplot3 [
					domain=0:4,
					domain y=0:4,
					samples=50,
					samples y=50,
					contour gnuplot={
						levels={-7, -5, -3, -1, 1, 3, 5, 7, 9, 11, 13},
						labels=false,
						draw color=black,
						handler/.style=smooth
					},
				] {.5*(2*x^2 + 2*x*y + 3*y^2) - 5*x - 5*y};
				
				\addplot[black,mark=none] coordinates {(0,0)  (10/7, 10/7)  (40/21, 20/21)  (290/147,50/49) (880/441,440/441) (6170/3087,1030/1029)};
			\end{axis}
		\end{tikzpicture}
	}
	
	\caption{Contour graph of the quadratic function and the first six values of \(\vec{x}\) produced by steepest decent for Example~\ref{ex:CG-SteepestDecent}}
	\label{fig:CG-steepestDecentExample-Iterations}
\end{figure}
\end{example}

%Conjugate Directions
The Conjugate Directions family of linear solvers, of which Conjugate Gradient is a member of, attempts to improve on the number of iterations needed by Steepest Decent.~\cite{Shewchuk:1994:IntroToCG}.
Note that, in Example~\ref{ex:CG-SteepestDecent}, the directions of \(\vec{r}_0\) and \(\vec{r}_2\) are the same and the directions of \(\vec{r}_1\) and \(\vec{r}_3\) are the same.
Thus, the same direction has to be traversed multiple times.
Additionally, note that the two sets of residual directions are perpendicular to each other.
Conjugate Directions attempts to improve on this, by making the search directions, \(\vec{d}_0, \vec{d}_1, \dots\), \(\mat{A}\)-orthogonal to each other and only moving \(\vec{x}\) once in each search direction.
Two vectors, \(\vec{u}, \vec{v}\) are \(\mat{A}\)-orthogonal, or conjugate, if \(\vec{u}^T\mat{A}\vec{v}=0\).
The requirement for Conjugate Directions is to make \(\vec{e}_{i+1}\) \(\mat{A}\)-orthogonal to \(\vec{d}_i\), where \(\vec{e}_i = \vec{x}_i - \mat{A}^{-1}\vec{b}\) is the error of \(\vec{x}_i\).
The computation of \(\alpha_i\) changes to find the minimal value along \(\vec{d}_i\) instead of \(\vec{r}_i\).
\[
	\alpha_i = \frac{\vec{d}_i^T\vec{r}_i}{\vec{d}_i^T\mat{A}\vec{d}_i}.
\]

%Conjugate Gradient proper
%TODO cite Intro to CG more aggressively
Conjugate Gradient is a form of Conjugate Directions where the residuals are made to be \(\mat{A}\)-orthogonal to each other~\cite{Shewchuk:1994:IntroToCG}.
This is done using the Conjugate Gram-Schmidt Process.
To do this, each search direction, \(\vec{d}_i\) is computed by taking \(\vec{r}_i\) and removing any components that are not \(\mat{A}\)-orthogonal to the previous \(\vec{d}\)'s.
So, let \(\vec{d}_0 = \vec{r}_0\) and for \(i > 0\) let
\[
	\vec{d}_i = \vec{r}_i + \sum_{k=0}^{i-1} \beta_{(i, k)}\vec{d}_k
\]
with \(\beta_{(i, k)}\) defined for \(i > k\).
Then, solving for \(\beta_{(i, k)}\) gives
\[
	\beta_{(i, k)} = -\frac{\vec{r}_i\cdot\mat{A}\vec{d}_i}{\vec{d}_j\cdot\mat{A}\vec{d}_j}.
\]
Note that each residual is orthogonal to the previous search directions, and thus the previous residuals.
So, it can be shown that \(\vec{r}_{i+1}\) is \(\mat{A}\)-orthogonal to all previous search directions, except \(\vec{d}_i\)~\cite{Shewchuk:1994:IntroToCG}.
Then, \(\beta_{(i, k)} = 0\) for \(i-1 \neq k\).
To simplify notation, let \(\beta_i = \beta_{(i, i-1)}\).
So, each new search direction can then be computed by
\[
	\vec{d}_i = \vec{r}_i + \beta_i\vec{d}_{i-1}.
\]
Algorithm~\ref{alg:bg-cg-alg-cg} shows the final Conjugate Gradient algorithm.

\begin{algorithm}[tb]
	\begin{algorithmic}
		\STATE \(\vec{r}_0 \gets \vec{b} - \mat{A}\vec{x}_0\)
		\STATE \(\vec{d}_0 \gets \vec{r}_0\)
		\FOR{\(i = 0, 1, \dots\) \textbf{until} \(\left\|\vec{r}_i\right\| \leq \epsilon\)}
			\STATE \(\alpha_i \gets \frac{\vec{r}_i\cdot\vec{r}_i}{\vec{d}_i\cdot \mat{A}\vec{d}_i}\)
			
			\STATE \(\vec{x}_{i+1} \gets \vec{x}_i + \alpha_i\vec{d}_i\)
			\STATE \(\vec{r}_{i+1} \gets \vec{r}_i + \alpha_i\mat{A}\vec{d}_i\)
			\STATE \(\beta_{i+1} \gets \frac{\vec{r}_{i+1}\cdot\vec{r}_{i+1}}{\vec{r}_i\cdot\vec{r}_i}\)
			\STATE \(\vec{r}_{i+1} + \beta_{i+1}\vec{d}_i\)
		\ENDFOR
	\end{algorithmic}
	
	\caption{Conjugate Gradient~\cite{Saad:2003:IterativeMethods}.}
	\label{alg:bg-cg-alg-cg}
\end{algorithm}


% 2 1
% 1 3
\begin{example}
\label{ex:CG-CG}
Consider the linear system used in Example~\ref{ex:CG-SteepestDecent} where
\[
	\mat{A} = \begin{bmatrix} 2 & 1 \\ 1 & 3 \end{bmatrix},
	\quad
	\vec{b} = \begin{bmatrix} 5 \\ 5 \end{bmatrix}.
\]
The result of applying Conjugate Gradient is
\begin{align*}
	 \vec{x}_0 &= \begin{bmatrix}0\\0\end{bmatrix}
	&\vec{r}_0 &= \begin{bmatrix}5\\5\end{bmatrix}
	& &
	&\vec{d}_0 &= \begin{bmatrix}5\\5\end{bmatrix}
	&\alpha_0  &= 2/7 \\
	%
	 \vec{x}_1 &= \begin{bmatrix}10/7\\10/7\end{bmatrix}
	&\vec{r}_1 &= \begin{bmatrix}5/7\\-5/7\end{bmatrix}
	&\beta_1   &= 1/49
	&\vec{d}_1 &= \begin{bmatrix}40/49\\-30/49\end{bmatrix}
	&\alpha_1  &= 7/10 \\
	%
	 \vec{x}_2 &= \begin{bmatrix}2\\1\end{bmatrix}
	&\vec{r}_2 &= \begin{bmatrix}0\\0\end{bmatrix}
\end{align*}
Note that after two iterations, \(\vec{x}\) reaches the exact solution, compared to the iterations of Steepest Decent in Example~\ref{ex:CG-SteepestDecent}.
Figure~\ref{fig:CG-CGExample-Iterations} shows the values of \(\vec{x}\) with the contour graph of the quadratic function. \qed
\begin{figure}
	\centering
	\ifdraft{}{
		\begin{tikzpicture}
			\begin{axis} [
				xlabel=\(x_1\),
				ylabel=\(x_2\),
				view={0}{90},
			]
				\addplot3 [
					domain=0:4,
					domain y=0:4,
					samples=50,
					samples y=50,
					contour gnuplot={
						levels={-7, -5, -3, -1, 1, 3, 5, 7, 9, 11, 13},
						labels=false,
						draw color=black,
						handler/.style=smooth
					},
				] {.5*(2*x^2 + 2*x*y + 3*y^2) - 5*x - 5*y};
				
				\addplot[black,mark=none] coordinates {(0,0)  (10/7, 10/7) (2, 1)};
			\end{axis}
		\end{tikzpicture}
	}
	
	\caption[Plotted iterations of Example~\ref{ex:CG-CG}.]{Contour graph of the quadratic function and each value of \(\vec{x}\) produced by Conjugate Gradient for Example~\ref{ex:CG-CG}.}
	\label{fig:CG-CGExample-Iterations}
\end{figure}
\end{example}

One way to improve the Conjugate Gradient method is to precondition the system~\cite{Saad:2003:IterativeMethods}.
Instead of solving the original system, \(\mat{A}\vec{x} = \vec{b}\), Conjugate Gradient solves \(\mat{M}^{-1}\left(\mat{A}\vec{x} - \vec{b}\right) = 0\) instead, where \(\mat{M}^{-1}\) is the preconditioner.
Note that \(\mat{M}\) should be similar to \(\mat{A}\), but \(\mat{M}^{-1}\) should be easier to compute than \(\mat{A}^{-1}\).
Algorithm~\ref{alg:bg-cg-alg-pcg} shows the preconditioned variant of the Conjugate Gradient.
\begin{algorithm}[tb]
	\begin{algorithmic}
		\STATE \(\vec{r}_0 \gets \vec{b} - \mat{A}\vec{x}_0\)
		\STATE \(\vec{z}_0 \gets \mat{M}^{-1}\vec{r}_0\)
		\STATE \(\vec{d}_0 \gets \vec{z}_0\)
		\FOR{\(i = 0, 1, \dots\) \textbf{until} \(\left\|\vec{r}_i\right\| \leq \epsilon\)}
			\STATE \(\alpha_i \gets \frac{\vec{r}_i\cdot\vec{z}_i}{\vec{d}_i\cdot\mat{A}\vec{d}_i}\)
			\STATE \(\vec{x}_{i+1} \gets \vec{x}_i + \alpha_i\vec{d}_i\)
			\STATE \(\vec{r}_{i+1} \gets \vec{r}_i + \alpha_i\mat{A}\vec{d}_i\)
			\STATE \(\vec{z}_{i+1} \gets \mat{M}^{-1}\vec{r}_{i+1}\)
			\STATE \(\beta_{i+1} \gets \frac{\vec{r}_{i+1}\cdot\vec{z}_{i+1}}{\vec{r}_i\cdot\vec{z}_i}\)
			\STATE \(\vec{d}_{i+1} \gets \vec{z}_{i+1} + \beta_{i+1}\vec{d}_i\)
		\ENDFOR
	\end{algorithmic}
	
	\caption{Preconditioned Conjugate Gradient~\cite{Saad:2003:IterativeMethods}.}
	\label{alg:bg-cg-alg-pcg}
\end{algorithm}