%TODO tables for days
%TODO conclusions from test results

Tables~\ref{tab:results-vec}, \ref{tab:results-val} and~\ref{tab:results-ind} show the compression results for compressing just the vector values, matrix values and matrix indices respectively.
Note that some compression strategies had multiple variations that were tested.
The compression of just one data structure fails to outperform the baseline implementation; Section~\ref{sec:results-bounds} discusses this further.

\begin{table}
	\centering
	\begin{tabular}{l|S|r|c}
		Compression & {GFLOP Rating} & Iterations & Compression Rate\\
		\hline
		Baseline & 15.3654 & 50 & 1:1 \\ %using mixed prec "all doubles" time
		Single Precision & 7.02301 & 115 & 1:2 \\
		Mixed Precision & & & \\ %b, x, d, Ad, r, z
		%TODO compute compression rates
		\hspace{3mm} \(\vec{d}\) & 5.18476 & 150 & 11:12 \\
		\hspace{3mm} \(\vec{b}, \vec{x}\) & 15.3701 & 50 & 5:6 \\
		\hspace{3mm} \(\vec{b}, \vec{x}, \mat{A}\vec{d}\) & 15.0428 & 51 & 3:4 \\
		\hspace{3mm} \(\vec{b}, \vec{x}, \vec{d}\) & 5.20832 & 150 & 3:4 \\ %using floats:b,x,d version
		\hspace{3mm} \(\vec{b}, \vec{x}, \vec{d}, \mat{A}\vec{d}\) & 5.23208 & 150 & 2:3 \\
		\hspace{3mm} \(\vec{b}, \vec{x}, \vec{d}, \mat{A}\vec{d}, \vec{z}\) & 6.95894 & 115 & 7:12 \\
		\hspace{3mm} \(\vec{b}, \vec{x}, \vec{d}, \vec{z}\) & 6.91343 & 115 & 2:3 \\
		\hspace{3mm} \(\vec{b}, \vec{x}, \vec{z}\) & 12.2146 & 64 & 3:4 \\
		%TODO consider adding half prec timings
		%TODO consider the single x fine compressions
		ZFP & & & \\
		\hspace{3mm} 1d & & & \\
			\hspace{6mm} 16 bits/value & 0.690138 & 51 & 1:4 \\
			\hspace{6mm} 32 bits/value & 0.39397 & 50 & 1:2 \\
		\hspace{3mm} 3d & & & \\
			\hspace{6mm} 16 bits/value & 2.72078 & 51 & 1:4 \\
			\hspace{6mm} 32 bits/value & 2.17639 & 50 & 1:2 \\
			\hspace{6mm} 48 bits/value & & & 3:4 \\
		SZ & & & \\
		%All timings have error setting of 1e-10 abs and 1e-10 rel
		%TODO figure out where to do an analysis on the optimal level of error bounding
		%TODO make a note that SZ comp rate is partially based on cache lines accessed
		\hspace{3mm} 7 values/block & 5.89138 & 57 & 8:7 \\
		\hspace{3mm} 8 values/block & 5.78711 & 57 & 1:1 to 2:1 \\
		\hspace{3mm} 12 values/block & 4.98536 & 57 & 2:3 to 4:3 \\
		\hspace{3mm} 15 values/block & 4.51594 & 57 & 8:15 to 16:15 \\
		\hspace{3mm} 16 values/block & 4.53536 & 57 & 1:2 to 3:2 \\
		\hspace{3mm} 24 values/block & 3.67748 & 57 & 1:3 to 4:3\\
		\hspace{3mm} 32 values/block & 3.1735 & 57 & 1:4 to 5:4\\
	\end{tabular}
	\caption{Results of Compressing Vector Values}
	\label{tab:results-vec}
\end{table}

%TODO finish filling in table
\begin{table}
	\centering
	\begin{tabular}{l|S|r|c}
		Compression & GFLOPs & Iterations & Compression Rate\\
		\hline
		Baseline & 15.3654 & 50 & 1:1 \\
		Single Precision & 12.6331 & 50 & 1:2 \\
		1-bit & 15.1743 & 50 & 1:64 \\ %cond op timing
		SZ & & & \\
		%TODO ensure description of SZ varients are clear
		%TODO get timings for other SZ versions
		%TODO need a comment mentioning how SZ's error bound is irrelavent for matVals & matInds
		\hspace{3mm} 1 mode & 13.5037 & 50 & \\
		\hspace{3mm} 2 modes & 13.8195 & 50 & \\
		%TODO get 3 modes setup
		%\hspace{3mm} 3 modes & & & \\
		ZFP & & & \\
		\hspace{3mm}High Level API & 0.817469 & 50 & \\
		\hspace{3mm}Low Level API & 0.960338 & 53 & \\
	\end{tabular}
	\caption{Results of Compressing Matrix Values.}
	\label{tab:results-val}
\end{table}
%TODO fill in table
\begin{table}
	\centering
	\begin{tabular}{l|S|r|c}
		Compression & GFLOPs & Iterations & Compression Rate\\
		\hline
		Baseline & 15.3654 & 50 & 1:1 \\
		SZ & 14.9322 & 50 & \\
		Gamma & 14.6553 & 50 & \\ %TODO consider using single alloc measure
		Delta & 14.3036 & 51 & \\
		Huffman & & & \\
		\hspace{3mm}Uncompressed First Index & & & \\
		%TODO explain meaning of "window"
			\hspace{6mm}4 bit window & 10.654 & 51 & \\
			\hspace{6mm}6 bit window & 10.7666 & 51 & \\
			\hspace{6mm}8 bit window & 10.8941 & 51 & \\
			\hspace{6mm}10 bit window & 10.8156 & & \\
			\hspace{6mm}12 bit window & 10.8158 & 51 & \\
		\hspace{3mm}Compressed First Index & & 51 & \\
			\hspace{6mm}4 bit window & 10.9359 & 51 & \\
			\hspace{6mm}8 bit window & 11.1134 & 51 & \\
			\hspace{6mm}16 bit window & 10.3323 & 51 & \\
		%TODO add single table timings
		%TODO add cont alloc timings (?)
		Op Code & & & \\
		%TODO add op code versions
	\end{tabular}
	\caption{Results of Compressing Matrix Indices}
	\label{tab:results-ind}
\end{table}
%TODO fill in table

%TODO add discussion of combined compression

\subsection{Performance Improvement Bounds}
\label{sec:results-bounds}
Note that Table~\ref{tab:results-val} shows 1 bit compression under performing the baseline implementation, even though it has a significant compression rate.
This demonstrates that compressing the matrix values alone in unable to improve performance.
For the vector values, note that the single precision implementation has a 2.3 times increase in iterations to convergence over the baseline implementation and that the GFLOPs rating of the single precision implementation is reduced by a factor of approximately 2.19 from the baseline implementation.
This hints that, even without increasing the number of Conjugate Gradient iterations, compressing the vectors will provide a limited improvement in performance at best.
This analysis is supported by the fact that no compression strategies that only compressed a single strategy was able to out perform the baseline implementation.

\subsection{Compiler Settings Analysis}
%TODO discuss why compiler settings don't have a significant affect on things

\subsection{Testing Environment}
%TODO test setup
Timings measured with a problem of size \(96^3\) with 60 processes on the walbert cluster.