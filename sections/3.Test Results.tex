Tables~\ref{tab:results-vec}, \ref{tab:results-val} and~\ref{tab:results-ind} show the compression results for compressing just the vector values, matrix values and matrix indices respectively.
These tables contain the rating measured by HPCG, the number of iterations needed for convergence and the compression rate based on the number of cache lines fetched, which may be different then the memory allocated.
Note that some compression strategies had multiple variations that were tested.
The compression of just one data structure fails to outperform the baseline implementation; Section~\ref{sec:results-bounds} discusses this further.

\begin{table}
	\centering
	\begin{tabular}{l|S|r|c}
		Compression & {GFLOP Rating} & Iterations & Compression Rate\\
		\hline
		Baseline & 15.3654 & 50 & 1:1 \\ %using mixed prec "all doubles" time
		Single Precision & 7.02301 & 115 & 1:2 \\
		Mixed Precision & & & \\ %b, x, d, Ad, r, z
		%TODO compute compression rates
		\hspace{3mm} \(\vec{d}\) & 5.18476 & 150 & 11:12 \\
		\hspace{3mm} \(\vec{b}, \vec{x}\) & 15.3701 & 50 & 5:6 \\
		\hspace{3mm} \(\vec{b}, \vec{x}, \mat{A}\vec{d}\) & 15.0428 & 51 & 3:4 \\
		\hspace{3mm} \(\vec{b}, \vec{x}, \vec{d}\) & 5.20832 & 150 & 3:4 \\ %using floats:b,x,d version
		\hspace{3mm} \(\vec{b}, \vec{x}, \vec{d}, \mat{A}\vec{d}\) & 5.23208 & 150 & 2:3 \\
		\hspace{3mm} \(\vec{b}, \vec{x}, \vec{d}, \mat{A}\vec{d}, \vec{z}\) & 6.95894 & 115 & 7:12 \\
		\hspace{3mm} \(\vec{b}, \vec{x}, \vec{d}, \vec{z}\) & 6.91343 & 115 & 2:3 \\
		\hspace{3mm} \(\vec{b}, \vec{x}, \vec{z}\) & 12.2146 & 64 & 3:4 \\
		%TODO consider adding half prec timings
		%TODO consider the single x fine compressions
		ZFP & & & \\
		\hspace{3mm} 1d & & & \\
			\hspace{6mm} 16 bits/value & 0.690138 & 51 & 1:4 \\
			\hspace{6mm} 32 bits/value & 0.39397 & 50 & 1:2 \\
		\hspace{3mm} 3d & & & \\
			\hspace{6mm} 16 bits/value & 2.72078 & 51 & 1:4 \\
			\hspace{6mm} 32 bits/value & 2.17639 & 50 & 1:2 \\
			\hspace{6mm} 48 bits/value & & & 3:4 \\
		SZ & & & \\
		%All timings have error setting of 1e-10 abs and 1e-10 rel
		%TODO figure out where to do an analysis on the optimal level of error bounding
		%TODO make a note that SZ comp rate is partially based on cache lines accessed
		\hspace{3mm} 7 values/block & 5.89138 & 57 & 8:7 \\
		\hspace{3mm} 8 values/block & 5.78711 & 57 & 1:1 to 2:1 \\
		\hspace{3mm} 12 values/block & 4.98536 & 57 & 2:3 to 4:3 \\
		\hspace{3mm} 15 values/block & 4.51594 & 57 & 8:15 to 16:15 \\
		\hspace{3mm} 16 values/block & 4.53536 & 57 & 1:2 to 3:2 \\
		\hspace{3mm} 24 values/block & 3.67748 & 57 & 1:3 to 4:3\\
		\hspace{3mm} 32 values/block & 3.1735 & 57 & 1:4 to 5:4\\
	\end{tabular}
	\caption{Results of Compressing Vector Values}
	\label{tab:results-vec}
\end{table}

%TODO finish filling in table
\begin{table}
	\centering
	\begin{tabular}{l|S|r|c}
		Compression & GFLOPs & Iterations & Compression Rate\\
		\hline
		Baseline & 15.3654 & 50 & 1:1 \\
		Single Precision & 12.6331 & 50 & 1:2 \\
		1-bit & 15.1743 & 50 & 1:64 \\ %cond op timing
		SZ & & & \\
		%TODO ensure description of SZ varients are clear
		%TODO get timings for other SZ versions
		%TODO need a comment mentioning how SZ's error bound is irrelavent for matVals & matInds
		\hspace{3mm} 1 mode & 13.5037 & 50 & \\
		\hspace{3mm} 2 modes & 13.8195 & 50 & \\
		%TODO get 3 modes setup
		%\hspace{3mm} 3 modes & & & \\
		ZFP & & & \\
		\hspace{3mm}High Level API & 0.817469 & 50 & \\
		\hspace{3mm}Low Level API & 0.960338 & 53 & \\
	\end{tabular}
	\caption{Results of Compressing Matrix Values.}
	\label{tab:results-val}
\end{table}
%TODO fill in table
\begin{table}
	\centering
	\begin{tabular}{l|S|r|c}
		Compression & GFLOPs & Iterations & Compression Rate\\
		\hline
		Baseline & 15.3654 & 50 & 1:1 \\
		SZ & 14.9322 & 50 & \\
		Gamma & 14.6553 & 50 & \\ %TODO consider using single alloc measure
		Delta & 14.3036 & 51 & \\
		Huffman & & & \\
		\hspace{3mm}Uncompressed First Index & & & \\
		%TODO explain meaning of "window"
			\hspace{6mm}4 bit window & 10.654 & 51 & \\
			\hspace{6mm}6 bit window & 10.7666 & 51 & \\
			\hspace{6mm}8 bit window & 10.8941 & 51 & \\
			\hspace{6mm}10 bit window & 10.8156 & & \\
			\hspace{6mm}12 bit window & 10.8158 & 51 & \\
		\hspace{3mm}Compressed First Index & & 51 & \\
			\hspace{6mm}4 bit window & 10.9359 & 51 & \\
			\hspace{6mm}8 bit window & 11.1134 & 51 & \\
			\hspace{6mm}16 bit window & 10.3323 & 51 & \\
		%TODO add single table timings
		%TODO add cont alloc timings (?)
		Op Code & & & \\
		%TODO add op code versions
	\end{tabular}
	\caption{Results of Compressing Matrix Indices}
	\label{tab:results-ind}
\end{table}
%TODO fill in table

Next, combined compression schemes were tried, using SZ and single precision compression for the matrix values and using SZ, gamma and delta compression for the matrix indices.
Table~\ref{tab:results-combined} shows the results of these combined schemes.
Like the single compression tables, this table contains the rating measure by HPCG, the number of iterations needed for convergence and the cache line compression rate.
The combined scheme with the best performance used SZ compression for both values and indices.
The only other approach that outperformed the baseline implementation used 32 bit compression for the values and gamma compression for the indices.

\begin{table}
	\centering
	\begin{tabular}{l|l|S|r|c}
		\multicolumn{2}{c|}{Compression} & & & \\
		Value & Index & GFLOPs & Iterations & Compression Rate\\
		\hline
		\multicolumn{2}{c|}{Baseline} & 15.3654 & 50 & 1:1 \\
		SZ & SZ & 18.9702 & 50 & \\
		SZ & Gamma & 13.661 & 51 & \\
		SZ & Delta & 10.9903 & 50 & \\
		32 bit & SZ & 14.1796 & 51 & \\
		32 bit & Gamma & 17.6676& 51 & \\
		32 bit & Delta & 12.56 & 51 & \\
	\end{tabular}
	\caption{Results of Combined Matrix Value and Index Compression Schemes.}
	\label{tab:results-combined}
\end{table}
%TODO fill in table

\subsection{Performance Improvement Bounds}
\label{sec:results-bounds}
Note that Table~\ref{tab:results-val} shows 1 bit compression under performing the baseline implementation, even though it has a significant compression rate.
This demonstrates that compressing the matrix values alone in unable to improve performance.
For the vector values, note that the single precision implementation has a 2.3 times increase in iterations to convergence over the baseline implementation and that the GFLOPs rating of the single precision implementation is reduced by a factor of approximately 2.19 from the baseline implementation.
This hints that, even without increasing the number of Conjugate Gradient iterations, compressing the vectors requires a compression rate better than 1:2 to provide much of an improvement in performance.
This analysis is supported by the fact that none of the compression strategies tried that only compressed a single strategy where able to out perform the baseline implementation.

\subsection{Vector Compression}
%TODO start off with general info
As shown in Table~\ref{tab:results-vec}, vector compression was not successfully used to improve performance.
Section~\ref{sec:results-bounds} discusses why performance improvement is likely limited.
However, vector compression is able to make improvements when combined with other compression.
%TODO get a table reference on this, prolly the table with vec-mix+vals-SZ+inds-SZ

ZFP had poor performance when compressing vector information.
Note that 1 dimensional ZFP compression has a 16 bit granularity, and 3 dimensional ZFP compression has a 1 bit granularity~\cite{Lindstrom:2014:zfp}.
These granularity restrictions and the resulting iterations needed were used to select the tested compression rates.

SZ compression has two main configurable settings, the number of values in each block and the error bound.
There were two measures of error that were considered, absolute error and pointwise relative error.
The performance was tested with both a single error being bounded and both errors being bounded.
Absolute error is the absolute value of the difference between predicted and actual.
The pointwise relative error is the absolute error divided by the actual value.
Table~\ref{tab:results-vec} contains results for various block sizes with both an absolute error bound of \(10^{-10}\) and a pointwise relative error bound of \(10^{-10}\).
Table~\ref{tab:results-vec-SZ} contains an comparison of various error bounds for a block size of 8 values per block.
%TODO get timings with another block size (12 val blocks?)
Note that an absolute bound of \(10^{-2}\) was unable to converge within 500 iterations.
%TODO is this note useful?

\begin{table}
	\centering
	\begin{tabular}{l|S|r}
		Error Bound & {GFLOP Rating} & Iterations \\
		\hline
		\(10^{-2}\) relative & 3.66859 & 69 \\
		\(10^{-6}\) relative & 5.7806 & 57 \\
		\(10^{-10}\) relative & 5.78711 & 57 \\
		\(10^{-14}\) relative & 5.81357 & 57 \\
		\(10^{-18}\) relative & 5.73277 & 57 \\
		\(10^{-2}\) absolute & {NA} & \(\geq 500\) \\
		\(10^{-6}\) absolute & 4.51827 & 57 \\
		\(10^{-10}\) absolute & 5.14058 & 57\\
		\(10^{-14}\) absolute & 5.64338 & 57 \\
		\(10^{-18}\) absolute & 5.81642 & 57 \\
		\(10^{-2}\) absolute and \(10^{-10}\) relative & 5.75538 & 57 \\
		\(10^{-10}\) absolute and \(10^{-2}\) relative & 5.22592 & 57 \\
		\(10^{-10}\) absolute and \(10^{-10}\) relative & 5.82527 & 57 \\
%TODO figure out if "or" tests should be used
%		\(10^{-2}\) absolute or \(10^{-10}\) relative & 5.77713 & 57 \\
%		\(10^{-10}\) absolute or \(10^{-2}\) relative & 5.16793 & 57 \\
%		\(10^{-10}\) absolute or \(10^{-10}\) relative & 5.80555 & 57 \\
	\end{tabular}
	\caption{Results of Compressing Vector Values with SZ Compression using Various Error Bounds.}
	\label{tab:results-vec-SZ}
\end{table}

\subsection{Compiler Settings Analysis}
%TODO discuss why compiler settings don't have a significant affect on things

\subsection{Testing Environment}
%TODO test setup
Timings measured with a problem of size \(96^3\) with 60 processes on the walbert cluster.