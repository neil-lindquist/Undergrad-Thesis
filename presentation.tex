\documentclass{beamer}

% presentation themeing
\usetheme{Singapore}
\beamertemplatenavigationsymbolsempty
\setbeamertemplate{bibliography item}{\insertbiblabel}

% output configuration
\usepackage{pgfpages}
%\setbeameroption{show notes on second screen=left}
\hypersetup{pdfpagemode=FullScreen}

% type setting packages
\usepackage{algorithmic}
\usepackage{amsmath}
\usepackage{siunitx}


% title slide
\title{\input{title.txt}}
\author{Neil Lindquist}
\date{April 16th, 2019}

\begin{document}
\begin{frame}
	\titlepage
\end{frame}

% Motivation
\begin{frame}
	\frametitle{Goal}
	\begin{itemize}
		\item Sparse linear systems used by many scientific computations
		\item Problems can be large, with over a million variables
		\item Arithmetic is faster than fetching data from memory
		%TODO mention cache
	\end{itemize}
\end{frame}

% Data access patterns/compression limitations
\begin{frame}[fragile]
	\frametitle{Solver Description}
	\begin{itemize}
		\item Preconditioned Conjugate Gradient was used for tests
		\begin{itemize}
			\item Preconditioned with a 3 level multigrid
			\item Symmetric Gauss-Seidel step smoother
		\end{itemize}
		\item Matrix store in CSR format
		\begin{itemize}
			\item Stores the column index and value for each nonzero entry
		\end{itemize}
		%TODO mention datastructures that can be compressed
	\end{itemize}
\end{frame}

%TODO consider describing test matrix

\begin{frame}
	\frametitle{Main Data Access Pattern}
	\begin{algorithmic}
		\FOR{row in rows}
			\FOR{nonzero entry in row}
				\STATE LOAD entry's value
				\STATE LOAD entry's column index
				\STATE LOAD vector value for column index
			\ENDFOR
			\STATE WRITE vector value for row
		\ENDFOR
	\end{algorithmic}
	
	\begin{itemize}
		\item need random vector reads
		\item need vector writes
		\item need both forward and backward iteration of matrix rows
	\end{itemize}
\end{frame}

% Model results

\begin{frame}
	\frametitle{Analytical Performance Model}
	\begin{itemize}
		\item System of equations that assumes no processor-level parallelism
		\item Solving for what conditions outperform the baseline gives:
	\end{itemize}
	\begin{align*}
%		\mathrm{vectEncode} <& 878.513 \\
		\mathrm{totalDecode} <& 32.5375 - 0.037037\cdot\mathrm{vectEncode} \\
		\mathrm{matBytes} <& 12.9664 - 0.398506\cdot\mathrm{totalDecode} \\
							&- 0.0147595\cdot\mathrm{vectEncode}\\
		\mathrm{vectBytes} <& 7.9998 + 8.27835\cdot(12 - \mathrm{matBytes}) \\
							&- 3.29897\cdot\mathrm{totalDecode}- 0.122184\cdot\mathrm{vectEncode}\\
	\end{align*}
\end{frame}

\begin{frame}
	\frametitle{Simulation Performance Model}
	\begin{itemize}
		\item Some processor level parallelism
		\item Conditions for outperforming the baseline:
	\end{itemize}
	
	%TODO consider switching to a graph
	\begin{tabular}{r|S|S|c}
      & {Matrix Index}  & {Matrix Value} & {Vector}\\
	{Bytes} & {Decode}        & {Decode}       & Encode and Decode \\
	\hline
	1 & 66 & 41 & \(4.75\geq 1.75\cdot\mathrm{decode}+\mathrm{encode}\)\\
	2 & 51 & 14 &\(4.75\geq 1.75\cdot\mathrm{decode}+\mathrm{encode}\)\\
	3 & 66 & 40 &\(2\geq 2\cdot\mathrm{decode}+\mathrm{encode}\)\\
	4 & 0 & 0 &\(0=\mathrm{decode}=\mathrm{encode}\) \\
	5 & {-} & 37 & \(4.75\geq 1.75\cdot\mathrm{decode}+\mathrm{encode}\)\\
	6 & {-} & 9 & Not Possible\\
	7 & {-} & 35 & \(0=\mathrm{decode}=\mathrm{encode}\)\\
	8 & {-} & 0 & \(0=\mathrm{decode}=\mathrm{encode}\)
\end{tabular}
\end{frame}

% Timing results

% Select Compression methods

% Conclusion


\end{document}
